%
%%%%%%%%%%%%%%%%%%%%%%%%%%%%%%
%%%%%%%%%%%%%%%%%%%%%%%%%%%%%%
%
%   ENVIRONMENTS
%
%%%%%%%%%%%%%%%%%%%%%%%%%%%%%%
%%%%%%%%%%%%%%%%%%%%%%%%%%%%%%
%
\newtheoremstyle{hunplain}%
  {\medskipamount}%         Space above
  {\medskipamount}%         Space below
  {\slshape}% Body font
  {}%         Indent amount (empty = no indent)
  {\bfseries}%Thm head font
  {.}%        Punctuation after thm head
  { }%        Space after thm head
  {\thmnumber{#2.\;}\thmname{#1}\thmnote{\normalfont\ [#3]}}
%
\newtheoremstyle{hundefinition}%
  {}%         Space above
  {}%         Space below
  {}%         Body font
  {}%         Indent amount (empty = no indent)
  {\bfseries}% Thm head font
  {.}%        Punctuation after thm head
  { }%        Space after thm head
  {\thmnumber{#2.\;}\thmname{#1}\thmnote{\normalfont\ [#3]}}
%
\newtheoremstyle{hunremark}%
  {}%         Space above
  {}%         Space below
  {}%         Body font
  {}%         Indent amount (empty = no indent)
  {\slshape}% Thm head font
  {.}%        Punctuation after thm head
  { }%        Space after thm head
  {\thmnumber{#2.\;}\thmname{#1}\thmnote{normalfont\ [#3]}}
%
\newtheoremstyle{hungyak}%
  {}%         Space above
  {}%         Space below
  {\normalfont}%         Body font
  {}%         Indent amount (empty = no indent)
  {\bfseries}% Thm head font
  {}%        Punctuation after thm head
  { }%        Space after thm head
  {\thmnumber{#2.\;}\thmnote{\bfseries\slshape (#3)}}
%
\theoremstyle{hunplain}
\newtheorem{thm}{Tétel}[section]
\newtheorem{cor}[thm]{Következmény}
\newtheorem{lm}[thm]{Lemma}
\newtheorem{prp}[thm]{Propozíció}
\newtheorem{clm}[thm]{Állítás}
\newtheorem{qu}[thm]{Kérdés}
\newtheorem{prb}[thm]{Probléma}
\newtheorem{fa}[thm]{Feladat}
%
\theoremstyle{hundefinition}
\newtheorem{df}[thm]{Definíció}
\newtheorem{exmp}[thm]{Példa}
\newtheorem{exc}[thm]{Gyakorlat}
%
\theoremstyle{hunremark}
\newtheorem{rem}[thm]{Megjegyzés}
%
\theoremstyle{hungyak}
\newtheorem{gyak}{} % no text
%
\numberwithin{equation}{section}
%
\newcommand{\texth}[1]{\text{\hun {#1}}}
\renewcommand{\proofname}{Bizonyítás}
%
%%%%%%%%%%%%%%%%%%%%%%%%%%%%%%
%
% Letters
%
\DeclareMathOperator{\BZ}{\mathbb{Z}} % integers
\DeclareMathOperator{\BQ}{\mathbb{Q}} % rationals
\DeclareMathOperator{\BR}{\mathbb{R}} % reals
\DeclareMathOperator{\BC}{\mathbb{C}} % complex numbers
\DeclareMathOperator{\BG}{\mathbb{G}} % Gaussian integers
\DeclareMathOperator{\BK}{\mathbb{K}} % Quaternions
\DeclareMathOperator{\BA}{\mathbb{A}} % Algebraic numbers
\DeclareMathOperator{\BF}{\mathbb{F}} % Finite fields
%
\newcommand{\Chi}{\mathbf{X}}
\newcommand{\vep}{\varepsilon}
%
% Classical
%
\DeclareMathOperator{\re}{\text{\rm Re}}
\DeclareMathOperator{\im}{\text{\rm Im}}
\DeclareMathOperator{\gr}{\text{\rm gr}}
\DeclareMathOperator{\tr}{\text{\rm tr}}
\DeclareMathOperator{\sg}{\text{\rm sg}} % signum, sign
\DeclareMathOperator{\tg}{{\rm tg}} % tangent
\DeclareMathOperator{\id}{\textsl{id}}     % The identity map
\DeclareMathOperator{\comp}{{\scriptstyle \circ}} % f o g
%
\DeclareMathOperator{\asszoc}{\sim} % associate (of a number)
\DeclareMathOperator{\ind}{\text{\rm ind}} % index
\DeclareMathOperator{\Ordo}{\mathit{O}}
%
\newcommand{\konj}[1]{\overline{#1}} % conjugate
\newcommand{\abs}[1]{\lvert #1\rvert} % absolute value
\newcommand{\card}[1]{\lvert #1\rvert} % cardinality
\newcommand{\norm}[1]{\lVert #1\rVert}
\newcommand{\scalar}[2]{\langle #1,#2\rangle} % scalar product
\newcommand{\wroot}[2]{\root\scriptstyle{#1}\of{#2}} % nicer looking root
%
% Set definition, logic
%
\newcommand{\sset}[1]{\{{#1}\}} % {...}
\DeclareMathOperator{\setMiddleSymbol}{:}
\newcommand{\set}[2]{\{#1\;\setMiddleSymbol\;%
 \nobreak\text{#2}\}}                      % { x | ... }
\newcommand{\bset}[2]{\big\{#1\;\setMiddleSymbol\;%
 \nobreak\text{#2}\big\}}                      % { x | ... }
\newcommand{\gen}[1]{\langle{#1}\rangle} % <...>
\DeclareMathOperator{\genMiddleSymbol}{\mid}
\newcommand{\Gen}[2]{\gen{\,#1\;\genMiddleSymbol\;%
 \nobreak\text{#2}}}                      % < x | ... >
%
\DeclareMathOperator{\union}{\cup}
\DeclareMathOperator{\inter}{\cap}  % intersection
\DeclareMathOperator{\meet}{\wedge}
\DeclareMathOperator{\join}{\vee}
\DeclareMathOperator{\bmeet}{\bigwedge}
\DeclareMathOperator{\bjoin}{\bigvee}
\DeclareMathOperator{\miff}{\Longleftrightarrow} 
                                           % math `if and only if'
\DeclareMathOperator{\mimplies}{\Longrightarrow} 
                                           % math `implies'
\DeclareMathOperator{\mimplied}{\Longleftarrow}  
                                           % math `implied by'
\DeclareMathOperator{\eqdef}{\stackrel{def}{=}}
                                           % defining equation
%
% Groups, rings
%
\newcommand{\GL}{\text{\rm GL}}
\newcommand{\SL}{\text{\rm SL}}
\newcommand{\PSL}{\text{\rm PSL}}
\newcommand{\PGL}{\text{\rm PGL}}
\newcommand{\Aff}{\text{\rm AGL}}
\newcommand{\Uni}{\text{\rm UT}}
\newcommand{\Tri}{\text{\rm T}}
\newcommand{\E}{\text{\rm E}}
\newcommand{\Ortog}{\text{\rm O}}
\newcommand{\SO}{\text{\rm SO}}
\newcommand{\Sph}{\text{\rm S}}
\newcommand{\U}{\text{\rm U}}
\newcommand{\SU}{\text{\rm SU}}
%
\DeclareMathOperator{\iso}{\cong}   % isomorphic
\DeclareMathOperator{\Ker}{\text{\rm Ker}} 
\DeclareMathOperator{\Hom}{\text{\rm Hom}}
\DeclareMathOperator{\Aut}{\text{\rm Aut}}
\DeclareMathOperator{\Inn}{\text{\rm Inn}}
\DeclareMathOperator{\Gal}{\text{\rm Gal}} % Galois group
\newcommand{\order}{o}
\DeclareMathOperator{\nm}{\triangleleft} % normal subgroup
\DeclareMathOperator{\sd}{\rtimes} % semidirect product
\DeclareMathOperator{\tensor}{\otimes}
\newcommand\ann{\text{\rm ann}}
\DeclareMathOperator*{\Oplus}{\bigoplus}
%
\long\def\ropzhtemplate#1#2#3{%
%#1 = kurzusnév
%#2 = sorszám, pl. II/7
%#3 = a feladat szövege
\newbox\tmpbox
\setbox\tmpbox\vbox to 0 pt{%

\centerline{\bf #1}

\centerline{\it #2.~röpdolgozat}

\bigskip

\noindent
\textbf{OLVASHATÓ név:}

\medskip

\noindent
\textbf{Neptun-kód:}

\medskip

\begin{gyak}
#3
\end{gyak}

\vskip 0pt plus 1 fil minus 1fil

}

\vbox to \textheight{

\copy\tmpbox

\vfill
\hrule
\vfill

\noindent
\begin{turn}{180}
    \box\tmpbox
\end{turn}
}}
%
\hyphenation{össze-adást négy-zet-össze-gét null-osz-tó
  null-osz-tó-men-tes null-osz-tó-men-tes-sé-get
  egy-ség-ele-mes nor-mál-osz-tók nor-mál-alak-ra
  nor-mál-alak-ját ge-ne-rá-tor-elem-nek bal-ide-ál
  bal-ide-ált bal-ide-ál-já-nak össze-adan-dó
  de-ter-mi-náns-osz-tó nem-tri-vi-á-lis meg-egyez-nek
  meg-egye-zik négy-zet-össze-ge-ként négy-zet-összeg
  össze-adás össze-adás-ból egy-ség-ele-mé-től bal-in-ver-ze
  gyű-rű-ele-mek egy-ség-elem fő-együtt-ha-tót
  kong-ru-en-cia-rend-szer-nek per-mu-tá-ció-cso-port
  nor-mál-osz-tó-nak per-mu-tá-ció-cso-por-tot
  cso-port-el-mé-le-tet mel-lék-osz-tá-lyok nor-mál-osz-tót
  nor-mál-osz-tó-it nor-mál-osz-tó-já-nak egy-ele-mű
  komp-lexus-szor-zás mind-erre cso-port-el-mé-let-ben
  cso-port-el-mé-le-ti mel-lék-osz-tály-ból nor-mál-osz-tó
  nor-mál-osz-tó-ra húsz-ele-mű maxi-má-lis
  rep-re-zen-táns-ele-mek egy-ség-elem-be in-dexük Suzuki
  Feit Thompson mel-lék-osz-tály prím-in-dexű
  rep-re-zen-táns-ele-met null-ele-me null-osz-tók fixen
  ge-ne-rá-tor-ele-me test-izo-mor-fiz-mus prím-ide-ál
  hal-maz-el-mé-le-ti tu-do-mány-ág két-ele-mű nor-mál-alak
  nor-mál-alak-ról fak-tor-al-geb-rái-val
  ge-ne-rá-tor-ele-mé-hez par-tí-ció-há-ló-já-nak
  po-li-nom-osz-tás-sal tit-kos-írást
  vissza-transz-po-nál-ják rész-struk-tú-ra
  rész-struk-tú-rá-it rész-al-geb-rá-ja
  nor-mál-osz-tó-há-ló-ja nyolc-ele-mű Morgan
  kva-ter-nió-test-ben négy-zet-elem át-in-dexel-ve köb-elem
  négy-ele-mű bal-ide-ál-ban fő-ide-ál-gyű-rűk
  négy-zet-össze-gek négy-zet-összeg négy-zet-össze-ge
  maxi-mu-ma sze-mi-di-rekt cso-port-al-geb-ra
  kon-ju-gált-osz-tály kon-ju-gált-osz-tá-lyok jobb-ide-ál
  bal-ide-ál bal-ide-ál-jai-ra maxi-ma-li-tá-sa
  de-fi-ní-ció-já-ból McKenzie sa-ját-altér
  kom-po-zí-ció-fak-to-ra-it kub-ok-ta-éder nor-mál-lán-ca
  test-át-ló Wiles null-ele-mből null-elem bal-ide-ál-ja
  osz-lo-pai-ban klón-el-mé-let mát-rixot mát-rixai
  mát-rixok-ra mát-rixok fény-év fény-éves sa-ját-al-te-re
  Eins-tein Eins-teint}
