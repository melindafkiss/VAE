%%%%%%%%%%%%%%%%%%%%%%%%%%%%%%
%%%%%%%%%%%%%%%%%%%%%%%%%%%%%%
%
%  Math macros
%
%%%%%%%%%%%%%%%%%%%%%%%%%%%%%%
%%%%%%%%%%%%%%%%%%%%%%%%%%%%%%
%
%
%%%%%%%%%%%%%%%%%%%%%%%%%%%%%%
%
%  Logical symbols
%
%%%%%%%%%%%%%%%%%%%%%%%%%%%%%%
%
\DeclareMathOperator{\satf}{\models}
\DeclareMathOperator*{\Oplus}{\bigoplus}
\DeclareMathOperator{\tensor}{\otimes}
\DeclareMathOperator{\miff}{\;\Longleftrightarrow\;} 
                                           % math `if and only if'
\DeclareMathOperator{\mimplies}{\;\Longrightarrow\;} 
                                           % math `implies'
\DeclareMathOperator{\mimplied}{\;\Longleftarrow\;}  
                                           % math `implied by'
\DeclareMathOperator{\setMiddleSymbol}{:}
\newcommand{\set}[2]{\{#1\;\setMiddleSymbol\;%
 \nobreak\text{#2}\}}                      % { x | ... }
\newcommand{\bset}[2]{\big\{#1\;\setMiddleSymbol\;%
 \nobreak\text{#2}\big\}}                      % { x | ... }
%\DeclareMathOperator{\id}{\textsl{id}}     % The identity map
\newcommand{\id}{\textsl{id}}     % The identity map
\DeclareMathOperator{\miffdef}{\stackrel{def}{\miff}}
                                           % defining iff
\DeclareMathOperator{\eqdef}{\stackrel{def}{=}}
                                           % defining equation

\newcommand{\labelledarrow}[1]{%
\stackrel{#1}{\longrightarrow}}
\DeclareMathOperator{\Mid}{\Bigg\vert}
\DeclareMathOperator{\bmid}{\Big\vert}
\DeclareMathOperator{\sbmid}{\big\vert}
%
%%%%%%%%%%%%%%%%%%%%%%%%%%%%%%
%
%%  Classical math
%
%%%%%%%%%%%%%%%%%%%%%%%%%%%%%%
%
\DeclareMathOperator{\BZ}{\mathbb{Z}} % integers
\DeclareMathOperator{\BQ}{\mathbb{Q}} % rationals
\DeclareMathOperator{\BR}{\mathbb{R}} % reals
\DeclareMathOperator{\BC}{\mathbb{C}} % complex numbers
\DeclareMathOperator{\BG}{\mathbb{G}} % Gaussian integers
\DeclareMathOperator{\BK}{\mathbb{K}} % Quaternions
\DeclareMathOperator{\BA}{\mathbb{A}} % Algebraic numbers
\DeclareMathOperator{\BF}{\mathbb{F}} % Finite fields
%
\DeclareMathOperator{\CC}{\mathcal{C}}
\DeclareMathOperator{\CD}{\mathcal{D}}
\DeclareMathOperator{\CG}{\mathcal{G}}
\DeclareMathOperator{\CH}{\mathcal{H}}
\DeclareMathOperator{\CK}{\mathcal{K}}
\DeclareMathOperator{\CL}{\mathcal{L}}
\DeclareMathOperator{\CP}{\mathcal{P}}
\DeclareMathOperator{\CS}{\mathcal{S}}
\DeclareMathOperator{\CX}{\mathcal{X}}
\DeclareMathOperator{\CV}{\mathcal{V}}
%
\DeclareMathOperator{\Var}{\mathsf{V}}
\DeclareMathOperator{\Ho}{\mathsf{H}}
\DeclareMathOperator{\Su}{\mathsf{S}}
\DeclareMathOperator{\Pd}{\mathsf{P}}
%
\DeclareMathOperator{\re}{\text{\rm Re}}% real part
\DeclareMathOperator{\im}{\text{\rm Im}}% imaginary part
\DeclareMathOperator{\Ker}{\text{\rm Ker}}
\DeclareMathOperator{\Hom}{\text{\rm Hom}}
\DeclareMathOperator{\Sub}{\text{\rm Sub}}
\DeclareMathOperator{\Con}{\text{\rm Con}}
\DeclareMathOperator{\Clo}{\text{\rm Clo}}
\DeclareMathOperator{\Pol}{\text{\rm Pol}}
%
\DeclareMathOperator{\meet}{\wedge}
\DeclareMathOperator{\join}{\vee}
\DeclareMathOperator{\bmeet}{\bigwedge}
\DeclareMathOperator{\bjoin}{\bigvee}
\DeclareMathOperator{\union}{\cup}
\DeclareMathOperator{\inter}{\cap}  % intersection
\DeclareMathOperator{\iso}{\cong}   % isomorphic
\DeclareMathOperator{\tg}{{\rm tg}}
\DeclareMathOperator{\compose}{{\scriptstyle \circ}} % f o g
\DeclareMathOperator{\nm}{\triangleleft}
\DeclareMathOperator{\sd}{\rtimes}
%
\newcommand{\wec}[1]{{\mathbf{#1}}}  % notation for vectors in algebras
%\newcommand{\wec}[1]{{\bar{#1}}} % which one do you prefer?
\newcommand{\wrel}[1]{\;#1\;}     % a\wrel{\alpha}b = a \alpha b
%              % this provides a flexible way to change the spacing
%
%\newcommand{\wrel}[1]{{\def\wrelarg{\qopname\relax{no}{#1}}\wrelarg}}
\newcommand{\abs}[1]{\lvert #1\rvert}
\newcommand{\card}[1]{\lvert #1\rvert}
\newcommand{\konj}[1]{\overline{#1}}
\newcommand{\norm}[1]{\lVert #1\rVert}
\newcommand{\asszoc}{\sim}
\newcommand{\order}{o}
\newcommand{\Order}{O}
\newcommand{\wroot}[2]{\root\scriptstyle{\vphantom{2^a} #1}\of{#2}}
\let\ts\relax
\newcommand{\gr}{\text{\rm gr}}
\newcommand{\GL}{\text{\rm GL}}
\newcommand{\SL}{\text{\rm SL}}
\newcommand{\PSL}{\text{\rm PSL}}
\newcommand{\PGL}{\text{\rm PGL}}
\newcommand{\Aff}{\text{\rm AGL}}
\newcommand{\Uni}{\text{\rm UT}}
\newcommand{\Tri}{\text{\rm T}}
\newcommand{\E}{\text{\rm E}}
\newcommand{\Ortog}{\text{\rm O}}
\newcommand{\SO}{\text{\rm SO}}
\newcommand{\Sph}{\text{\rm S}}
%\newcommand{\U}{\text{\rm U}}
\newcommand{\SU}{\text{\rm SU}}
%\newcommand{\GF}{\text{\rm GF}} % old-fashioned?
\newcommand{\GF}[1]{\BF_{{#1}}}
%
%%%%%%%%%%%%%%%%%%%%%%%%%%%%%%
%%%
%%%  \m is used to produce models (algebras) as follows:
%%%  typing `\m a'  produces in bold face, A.
%%%
%%%%%%%%%%%%%%%%%%%%%%%%%%%%%%
%
\newcommand{\m}[1]{{\mathbf{\uppercase{#1}}}}
%
%%%%%%%%%%%%%%%%%%%%%%%%%%%%%%
%
%  Abbreviations
%
%%%%%%%%%%%%%%%%%%%%%%%%%%%%%%
%
\newcommand{\lb}{\langle} % <...>
\newcommand{\rb}{\rangle}
\newcommand\gen[1]{\lb#1\rb}
\newcommand\scalar[2]{\lb #1,#2\rb}
\DeclareMathOperator{\genMiddleSymbol}{\mid}
\newcommand{\Gen}[2]{\gen{\,#1\;\genMiddleSymbol\;%
 \nobreak\text{#2}}}                      % < x | ... >
%
\newcommand{\ul}{\underline}
\newcommand{\ol}{\overline}
%
\newcommand{\vep}{\varepsilon}
%
%%%%%%%%%%%%%%%%%%%%%%%%%%%%%%%%%%%%%%%%%%%%%%%%%%%%%%%
\newcommand\texth[1]{\text{\hun {#1}}}
\newcommand\Aut{{\text{{\rm Aut}}}}
\newcommand\Inn{{\text{{\rm Inn}}}}
\newcommand\tr{\text{\rm tr}}
\newcommand\sg{\text{\rm sg}}
\newcommand\ann{\text{\rm ann}}
\newcommand\Chi{\mathbf{X}}
\newcommand\wector[1]{\overrightarrow{#1}}
\newcommand\hph{\hphantom{-}}
\newdimen\wstrutht
\wstrutht=\ht\strutbox\advance\wstrutht\dp\strutbox
\newcommand\wstrut{\vrule width 0pt height\wstrutht
  depth\dp\strutbox}
\newcommand\hphm{\hphantom{-}}
%\newcommand\wstrut{\vphantom{$2^{A^A}$}}

%Cells in a Horner scheme
\newbox\Hcellbox
\setbox\Hcellbox\hbox{$-4$}
\newdimen\Hcellwidth
\Hcellwidth=\wd\Hcellbox
\newcommand\Hc[1]{\hbox to\Hcellwidth{\hfill{$#1$}\hfill}}

\definecolor{wfgreen}{rgb}{.132,.545,.132}
\definecolor{wfyellow}{rgb}{.8,.8,.0}
\def\wmathred#1{\only<presentation>{{\setbeamercolor{math text}{fg=red}{{#1}}}}\only<article>{#1}}
\def\wcyan#1{\only<presentation>{{\color{cyan}{{#1}}}}\only<article>{{#1}}}
\def\wyellow#1{\only<presentation>{{\color{wfyellow}{{#1}}}}\only<article>{{#1}}}
\def\wmathyellow#1{\only<presentation>{{\setbeamercolor{math text}{fg=yellow}{{#1}}}}\only<article>{#1}}
\def\wred#1{\only<presentation>{{\color{red}{{#1}}}}\only<article>{{\underline{#1}}}}
\def\wblack#1{\only<presentation>{{\color{black}{{#1}}}}\only<article>{{\underline{#1}}}}
\def\wblue#1{\only<presentation>{{\color{blue}{{#1}}}}\only<article>{{\underline{#1}}}}
\def\itemn#1{\only<presentation>{{\color{blue}{{#1}}}}\only<article>{{#1}}}
\only<presentation>{\setbeamercolor{math text}{fg=wfgreen}}


\endinput

